\documentclass{article}
\usepackage[utf8]{inputenc}

\usepackage{amsthm}
\usepackage{amssymb}
\usepackage{amsmath}
\usepackage{color}

\usepackage{geometry}
\geometry{a4paper,
  total={170mm,220mm},
  marginparwidth=80mm,
left=5mm,
right=85mm,
top=20mm,
}

% For algorithms
\usepackage{algorithm}
\usepackage{algorithmic}


\title{Active learning using muffler and partitions}
\author{Akshay Balsubramani, Yoav Freund, Shay Moran}
\date{December 2016}

\newtheorem{theorem}{Theorem}[section]
\newtheorem{corollary}{Corollary}[theorem]
\newtheorem{lemma}[theorem]{Lemma}
\newtheorem{assumption}[theorem]{assumption}
\newtheorem{definition}[theorem]{Definition}

\newcommand{\err}{\mbox{err}}
\newcommand{\X}{{\cal X}}
\newcommand{\Y}{{\cal Y}}
\newcommand{\x}{\vec{x}}
\newcommand{\y}{\vec{y}}
\newcommand{\vv}{\vec{v}}
\newcommand{\cc}{\vec{c}}

\newcommand{\K}{{\cal K}}
\newcommand{\restrictedto}{\triangleright}
\renewcommand{\SS}{{\cal S}}
\newcommand{\outcome}{z}
\newcommand{\empoutcome}{\hat{\outcome}}
\newcommand{\polarity}{p}
\newcommand{\diameter}{D}
\newcommand{\weight}{\omega}

\newcommand{\bias}{\text{bias}}
\newcommand{\sign}{\text{sign}}
\newcommand{\bayes}{\beta}
\newcommand{\new}[1]{\textcolor{red}{#1}}

\newcommand{\comment}[3]{\marginpar{\textcolor{#2}{#1: #3}}}
%\newcommand{\comment}[3]{}
\newcommand{\shay}[1]{\comment{Shay}{red}{#1}}
\newcommand{\yoav}[1]{\comment{Yoav}{blue}{#1}}
\newcommand{\akshay}[1]{\comment{Akhay}{magenta}{#1}}

\begin{document}

\maketitle

\section{Introduction}

As a first step towards using Muffler for active learning, we describe
a setup in which Muffler converges to the Bayes optimal rule.

We operate in a restricted context which emulates the kNN 
convergence rate analysis of Chaudhuri and Dasgupta.

\section{Problem setup}

The input space $\X$ is a finite set of points in $R^d$.
The label of each point is drawn according to a conditional
distribution over $\{-1,+1\}$. Instead of the conditional
probability, we use the conditional expectation
$\outcome(\x) \doteq E[Y | X=\x]$.

The goal is to achieve error close to the Bayes optimal rule, which is
\[
\beta(\x) =
\begin{cases}
 -1 & \outcome(\x)<0\\
 0 & \outcome(\x)=0\\
 +1 & \outcome(\x)>0
\end{cases}
\]

\subsection{Smoothness assumptions}

In order to ensure that the problem is learnable we restrict the
distribution over $R^d \times \{-1,+1\}$.
\newcommand{\borderdist}{{\cal d}}
We define the distance of $\x \in R^d$ to the boundary as
\[
\borderdist(\x) \doteq \min \left\{ d(\x,\y)\;\; \text {s. t.}\;\; \beta(\x) \neq
\beta(\y) \right\}
\]
We say that the border is vanishingly small if the probability of the
points for which $\borderdist(x) < \epsilon$ decreases to zero at
least as fast as $O(1/\epsilon)$

We also define The support 




We define the 


\section{Specialists}

The base rules we use are ``specialists'', a specialist is defined by
a set $S \subset R^D$ and a polarity $\polarity \in \{-,+\}$. The
specialist $S_+$ maps points in $S$ to $+1$ and points outside $S$ to
$0$. Similarly, the specialist $S_-$ maps points in $S$ to $-1$ and
points outside $S$ to $0$.

We define two properties for each specialist: a weight $\weight(S)$
and a diameter $\diameter(D)$. Both depend just on the set $S$ and not
on the polarity. The {\em weight} of a specialist to be the fraction $\X$
that is in $S$:
\[
\weight(S) \doteq \frac{|S \cap \X|}{|S|}
\]
The diameter of a specialist is defined using euclidean distance:
\[
\diameter(S) = \text{sup} \left\{ \|\x-\y \|_2 \;\;\text{ s.t.}\;\;   \x,\y
\in S \right\}
\]

In this paper we restrict ourselves to cubic specialists,
i.e. specialists whose set $S$ is defined by a position $\x in R^d$
and a size $r \in R^+$. More precisely we define a cubic set as
\[
\text{Cube}\left(\x,r\right) \doteq
\left\{\y \in R^d\;\; \text{s.t.}\;\; \forall i \in 1,\ldots,d\;\;\; x_i
\leq y_i < x_i+r   \right\}
\]

Clearly, the diameter of the cube $\text{Cube}(\x,r)$ is $\sqrt{d} r$

\section{

\section{Grid Sequence}

The 

The specialists are grouped in a sequence of sets called {\em levels}
$\{\SS_i\}_{i=1}^\infty$.  Each level defines a {\em partition} of
$\X$. In other words a collection of disjoint sets whose union covers
$\X$.

One of the sets in each level is called the {\em outliers set} and is
denoted by $O_i$, this is a set on which all the specialists of level
$i$ abstain.

The other {\em regular} sets are denoted by $S \in \SS_i$.
each regular set defines two specialists: a positive one and a negative one.

The diameters and weights of the sets obey the following conditions:
\begin{enumerate}
\item We define the $\diameter_i$ to be the maximal diameter over the
  sets of the corresponding partition. $$\diameter_i \doteq \max
  \left\{ \diameter(S),\;\;S \in \SS_i \right\} $$
\item We assume that the weights of each set in a partition to be
  approximately equal. Specifically, we associate with level $i$ a
  {\em weight} $\weight_i$ such that $$ \forall S \in \SS_i:\;\;
  \weight_i/2 \leq \weight(S) \leq \weight_i $$
\item The diameter of the outlier set $O_i$ unconstrained and it's
  weight is denoted $\delta_i$.
\end{enumerate}

As $i$ increases, the sets become smaller. As $\X$ is finite, for a
sufficiently large $i$ the outliers set is empty and each of the sets
in $\SS_i$ contains exactly one point. The rate at which
$\diameter_i, \weight_i$ and $\delta_i$ decrease characterizes the
convergence rate of this system of specialists.

We define the {\em bias} of a specialist $S \in \SS$ as the 
conditional expectations over the support of $S$.
$$
\bias(B) \doteq \frac{1}{|S|} \sum_{\vv \in S} \outcome(\vv)
$$

\section{Idealized algorithm}

The information given to Muffler is a set of constraints over
$\bias(B)$. These constraints are of the form.

Our analysis has two steps. First, we use an Idealized algorithm,
which has access to e We
use the idealized algorithm to characterize the convergence of
Muffler to the Bayes optimal rule as $k$, the minimal weight of the balls
decreases to one.

After the analysis of the idealized algorithm, we will define and
analyze the actual active learning algorithm which has access only to
queried labels

\begin{algorithm}[]
   \caption{An Idealized Muffler Learning Algorithm}
   \label{alg:activealg}
\begin{algorithmic}
   \FOR{$k= |\X|,|\X|-1,\ldots,1$ }
   \STATE \textbf{\textcolor{blue}{Compute Constraints}}
   \STATE{Define the constraint set as
     $L_k = \{\outcome(B)=\bar{\outcome}(B) | B \in S_k$}
   \STATE \textbf{\textcolor{blue}{Solve}}
   \STATE{Find the optimal muffler solution for the equality
     constraints $\{L_k\}$ and find, for each $\x \in \X$, 
   the prediction $f_k(\x) = \sign(\outcome(x))$; }
   \ENDFOR
\end{algorithmic}
\end{algorithm}

\section{Neighborhood size}

The following definition characterizes the rate of convergence size of the neighborhood of a
point $\x$ that is consistent with the Bayes prediction on $\x$
\begin{definition}[Neighborhood support]
Let $p(\x)$ be the Bayes optimal prediction at the point $\x$. In
other words $p(\x) = \sign(\outcome(\x))$. We define the $k$ neighborhood
support  of the  point $\x$ to be the minimal value of $p(\x)\bias(B)$
over all balls $B$ such that $\x \in B$ and $|B|\leq k$. We use the
notation $\gamma_k(\x)$ to denote this quantity.
\end{definition}

This definition is what we need for our analysis. We will show a
non-trivial lower bound on $\gamma_k(\x)$ for the case where the
points $\x$ are drawn from a density distribution over $R^n$ with a
Lipshitz condition and the conditional probability $\outcome(\x)$ also
satisfies a lipshitz condition.


The second definition is a technical condition that we need for our
proofs that (I believe) holds for $\X$ which is drawn from a density
distribution over $R^n$
\begin{definition}[Decreasing support]
The set $\X$ provides decreasing support for the set of balls if for
any $\x \in \X$ and for any ball $B$ that covers $\x$ there exists
another ball $B' \subset B$ that also covers $\x$ such that $|B'| = |B|-1$.
\end{definition}

\shay{I think the above definition is satisfied by every finite set:
Let $\x \in \X$, and let $B$ be a ball covering $\x$. 
It seems possible to modify the ball ever-so-slightly, 
such that (i) its intersection with $\X$ is unchanged,  and (ii) the 
distances of the points in $\X$ from its center are distinct.
Now, $B'$ is obtained by decreasing the radius of $B$ until
the furthest point in it from its center is removed.  Does it make sense?}
\yoav{Yes, sounds good, the only small refinement is that you need to
  show that there is always a way to do this which does not remove the
  special point $\x$.}

\section{Passive learning}

We first analyze the convergence rate of a passive learning algorithm.
This algorithm is Muffler using balls as specialists. More
specifically, we assume that the input space $\X \new{\subseteq}  R^n$ is finite
and that the training set is generated by repeatedly drawing a point
$\x$ from $\X$ uniformly at random and labeling the point according
to the binary distribution of $y$ conditioned on $\X$. Note that two
examples with $\x$ as the input can have different labels.

It is trivial that this algorithm will asymptotically converge to the
Bayes optimal rule. The rate of convergence at a point $\x$ depends on 
the rate at which $\gamma_k(\x)$ increases as $k$ decreases.
%
%\shay{I am not sure why we assume a finite $\X$. 
%Shouldn't we use the the standard assumptions usually made in the kNN setting?}
%
%\yoav{I am doing it because it makes the analysis simpler, finite
%  number of instances, finite number of ball-induced subsets etc. On
%  the other hand, we lose nothing, assuming non of our bounds depends
%  on the size of $\X$.}
%  
%  \shay

\pagebreak
\section{Active Learning}

In this section we provide a variant of the Muffler algorithm that
requires the labeling of far fewer examples than passive sampling.

We review and define some notation we use to describe the active learning algorithm: 

\begin{itemize}
\item $\X$ is the set of unlabeled data which defines the
  universe for the problem.
\item $\K_i \subset \X$ is the set of ``known unknowns'' after iteration $i$, in
  other words the set of ``unsafe'' points. $\K_0=\emptyset$.
\shay{I am a bit confused. Shouldn't at the beginning the set of ``unsafe''
points be everything? At the beginning we have no knowledge and therefore can not be ``sure''
of anything\ldots}
  
\item For any set $A\subseteq \X$, denote $Q(A,n)$ to be a set of
  $n$ labeled examples $(\x,y)$ generated as follows. $\x$ chosen
  independently at random from the uniform distibution over
  $A$. The label $y$ is then chosen independently at random according
  to the distribution $\outcome(\x)$. We say that the
  labels in $Q(A,n)$ {\em support} a specialist $B \in S$ if $B
  \subseteq A$.
  \shay{The last definition does not make sense to me:
  $Q(A,n)$ is a distribution over labeled samples of size $n$.
  Therefore, it does not make sense to consider the ``labels in $Q(A,n)$''.
  We may consider the labels of some sample $R\sim Q(A,n)$,
  but then the wording "labels in $Q(A,n)$ support a specialist $B$\ldots"
  does not make sense: it depends only on the distribution $Q(A,n)$, and
  not on a particular sample from it.
  }
  \yoav{$Q(A,n)$ is a set of $n$ labeled examples. In other words, it
    is a random set, not a distribution over labeled examples.}
  \shay{I meant the same thing: by a sample I mean a sequence
  of labeled examples. Thus, $Q(A,n)$ is a distribution over samples.}
  \shay{Perhaps this will explain better my previous comment:
  Fix some set $A$, and let $R$ be a random sample drawn according to $Q(A,n)$.
 I see two possible interpretations for the definition "labels in $Q(A,n)$ support a specialist $B$'':
 (i) All the points in $B$ appear as examples in $R$,
 (ii) $B\subseteq A$.
 I guess you mean the second one, but I find the wording confusing. }
\item We define the restriction of a specialist set $S$ to a set of
  examples $\K \subset \X$ as:
  $$
  S \restrictedto \K \doteq \left\{ B \cap \K : B \in S \right\}
  $$
  %\shay{Wouldn't $S \restrictedto \K \doteq \left\{ B \cap \K : B \in S \right\}$ work?}
\end{itemize}


\pagebreak
Yoav: I moved the comments out of the algorithm because side comments
don't seem to work with ``algorithm''

   \akshay{This can be changed when we define a specialist hierarchy;
     as it stands, it seems to me that $\SS_0 = S$ throughout. }
   \yoav{$S \restrictedto \K_{i-1}$ is not equal to $S$ in most
     cases. For example suppose that  $\K_{i-1}$ is the intersection
     of a ball will $\X$. Then $S \restrictedto \K_{i-1}$ contains all
     intersections of this ball with some other ball, and an
     intersection of two balls is not a ball (in general).}

    \akshay{I think here $L_i$ should just be the set in brackets, not
     ``$\restrictedto B$".}
   \yoav{In the sense of the prediction error, this does not matter,
     because outside of $B$ the prediction is zero. However we want to
     use the conditional probability, so we need to divide by the size
     of $L_i$ and $L_i$ must contain only the points inside $B$}
   \akshay{I understand this to be the set of predictions $\subseteq [-1,1]$ on $\x$ that are possible minimax optimal muffled predictions.} 
   \yoav{That is correct, and it is important because the optimal
     solution is constant on each atom of the algebra defined by the
     balls, while an arbitrary solution does not need to be a constant.}
   \akshay{I believe you mean $f(\x)$ as defined above. In this case, 
   the unsafe set I defined in the earlier writeup (by analogy with CAL-style algorithms; call it $N_i$) is a superset of $\K_i$, 
   because $N_i$ includes any point $\x$ which can be hedged, not just
   the ones which can have score $0$. }
   \yoav{I will have to think moew about this, but I think you are right. The
     question is what can be said about the set of hedged examples of
     the idealized algorithm.}

\pagebreak
\begin{algorithm}[]
   \caption{An Active Learning Algorithm}
   \label{alg:activealg}
\begin{algorithmic}
   \STATE \textbf{\textcolor{blue}{Initialize}}   $\K_0=\emptyset$, $\SS_0 = S$,
   \FOR{$i = 1,\ldots$ }
   \STATE \textbf{\textcolor{blue}{Query}}: Generate two new sets of labeled examples, $U_i$ and $F_i$. 
   \STATE{$U_i = Q(\X,n)$}
   \IF{$|\K_{i-1}|>0$}
   \STATE{$F_i = Q(\K_{i-1},n)$ and $\SS_i = \SS_{i-1} \cup (S \restrictedto \K_{i-1})$}
   \ELSE
   \STATE{$F_i = \emptyset,\; \SS_i=\SS_{i-1}$}
   \ENDIF
   \STATE \textbf{\textcolor{blue}{combine labeled data}}
   \FOR{$B \in \SS_i$}
   \STATE Define the labeled set for $B$ as:
   $$ L_i(B) = \left[ \cup_{j=1}^i U_j \bigcup \cup_{j \leq i: \{B \subseteq \K_j\}} F_j)\right] \restrictedto B$$
   \ENDFOR
   \STATE \textbf{\textcolor{blue}{Compute Constraints}}
   \FOR{$B \in \SS_i$}
   \STATE{Compute the empirical mean label for $B$:
     $$\empoutcome(B) = \frac{\sum_{(\x,y) \in L_i(B)}y}{|L_i(B)|}$$}
   \STATE{Compute the error bound for $B$:
     $$\Delta(B) = \sqrt{\frac{\ln 2i^2/\delta + (d+1) \ln
         |L_i(B)|}{|L_i(B)|/2}} $$
   where $d$ is the dimension of the space and $\delta$ is the
   confidence parameter.}
   \STATE Set the confidence interval for $B$ to be 
   of $C_i(B)=[\empoutcome(B) \pm \Delta(B)] \cap [-1,+1]$ 
   \ENDFOR
   \STATE \textbf{\textcolor{blue}{Solve}}
   \STATE{Solve the linear constraints $\{L_i(B)\}$ and find, for each $\x \in \X$, 
   the segment of feasible solutions $f(\x)$; 
   Define the new set of known unknowns to be
   $$ \K_i = \{\x | 0 \in f(\x)\} $$}
     
   \ENDFOR
\end{algorithmic}
\end{algorithm}

The analysis of the algorithm consists of two steps.
First, we analyze an idealized algorithm
which operates under the assumption that exact conditional
probabilities are available. Then, in the second step, we remove the
assumption of exact conditional probabilities, replace it with
estimates based on a sample, and analyze the sample based algorithm.

We define the set of all balls whose support size is at most $k$ by $S_k$.

The parametrized algorithm $I(k)$ uses $S_k$ as the set of specialists. 
It receives as input the exact bias, ``$\bias(B)$'', for all $B \in S_r$.

The algorithm solves the min-max weighting of the rules $\sigma^*(r)$.

The main lemmas we use to analyze this algorithm are:
\begin{lemma}[small set domination]
  Let $S_k(\x)$ contain all $B \in S_k$ such that $\x \in B$, and
  suppose $\gamma_k(\x)>0$. Then the prediction of the idealized
  muffler on $\x$ is correct.
\end{lemma}

Which follows from the following

\begin{lemma}[Subset Domination]
Suppose $B_1 \subset B_2$ are two balls that cover a point $\x$ and
suppose that the constraints have opposite polarity: $\bias(B_1)>a>
>0>b>\bias(B_2)$ or $\bias(B_1)<a<0<b<\bias(B_2)$.  Then the
prediction of Muffler on $\x$ remains uncanged if the constraint
corresponding to $B_2$ is eliminated.
\end{lemma}

{\bf Sketch of Proof:}
Assume w.l.o.g. that the first condition holds:
$\bias(B_1)>a>0>b>\bias(B_2)$. As $\X$ is finite these conditions can
be written as:
$$
\sum_{i=1}^n \bias(\x_i) > na
\text{ and }
\sum_{i=1}^m \bias(\x_i) < mb
$$
where $m>n$.
But that is equivalent to the conditions
$$
\sum_{i=1}^n \bias(\x_i) > na
\text{ and }
\sum_{i=n+1}^m \bias(\x_i) < mb-na
$$
In other words, if $i \leq n$, which means that $\x_i \in B_1$, then
the constraint on $B_2$ is irrelevant

\yoav{I realize that this argument is incomplete, because the
  influence on $b$ can be through other balls. However, I believe that
if {\em all} balls of size at most $k$ containing $\x$ made the
correct predictions, than they mask all larger balls.}

\akshay{I agree that this is good for the analysis. However I am not sure how to prove it. \\ \\
Here is one idea: this will happen if $B_2$ is small enough (suppose e.g. 
that the balls have support size $2^i$ for integer $i$; then each specialist's prediction is  
weighted proportionally to this $2^i$ by the muffler aggregation, 
so if not too many balls of the same size predict at $\x$, 
then the optimal prediction on $\x$ is essentially determined by the smallest specialist predicting on it). 
}

An adaptive version of the VC bound:
\begin{lemma}
Suppose $L$ is a set of labeled examples chosen IID from a
joint distribution over $\X \times Y$. Suppose $S$ is a set of
specialists of VC dimension $d$.
Let $m(B)=|L \cap B|$ denote the number of elements of $L$ that are
inside $B \in S$, let $\outcome(B)$ be the mean of the labels in $\X \cap B$ and
let $\empoutcome(B)$ be the empirical average of the labels in $L \cap B$. 

Then for all $\delta>0$:
$$
Pr\left[ \exists B\in S \text{ s.t. }  |\outcome(B) - \empoutcome(B)| >
  \sqrt{\frac{\ln 2/\delta+d \ln m(B)}{m(B)/2}} \right] < \delta
$$
\end{lemma}




\end{document}
