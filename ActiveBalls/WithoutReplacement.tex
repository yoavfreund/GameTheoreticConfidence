\subsection{sampling without replacement}

% Transition probabilities
\newcommand{\p}{p}
\newcommand{\q}{q_{t,i}}

%$$\p=k/n$$
%$$\q=\frac{k-i}{n-t}$$

% RVs
\newcommand{\cnt}[2]{k^{#1}_{#2}}

%  bin probabilities
\newcommand{\prob}[2]{#1_{#2}}

% tail probabilities
\newcommand{\tail}[3]{{\mathbf #1}^{#2}_{#3}}

% mode
\newcommand{\mode}[1]{\mbox{mode}\left(#1\right)} 
\newcommand{\fmode}[1]{Z^{#1}_t}


Before proving Theorem~\ref{thm:Bias-Convergence} we prove a 
lemma regarding the concentration of sampling without replacement.

Let $a_1,\ldots,a_n$ be a sequence where $a_i \in
\{0,1\}$. We denote the number of 1's by $k=\sum_{i=1}^n a_i$.

We compare sampling $n$ times from $a_1^n$ $n$ with and without
replacement. We denote by $\cnt{P}{t}$ the random variable
corresponding to the number of 1's after sampling $t$ times {\em with
  replacement}, and by $\cnt{Q}{t}$ the corresponding random
variable when the sampling is done {\em without replacement}.

We deonte the marginal probability that $\cnt{P}{t}=i$ by
$\prob{P}{t,i}$ and the probability that $\cnt{Q}{t}=i$ by
$\prob{Q}{t,i}$.

Both $\prob{P}{t,i}$ and $\prob{Q}{t,i}$ have closed form formulas:
\begin{equation}\label{eqn:P-closed-form}
\prob{P}{t,i} = {t \choose i} \left(\frac{k}{n}\right)^i \left(1-\frac{k}{n}\right)^{t-i}
\end{equation}

\begin{equation}\label{eqn:Q-closed-form}
\prob{Q}{t,i}=\frac{{k \choose i}{n-k \choose t-i}}{{n \choose t}}
\end{equation}

We denote the probability of sampling a 1 when sampling with
replacement by $\p=k/n$ and the probability of sampling a 1 when
sampling with replacement at step $t$ given that the number of 1's
sampled so far is $i$ by $\q=\frac{k-i}{n-t}$

Note that if $i < \p t$ then
\[
\q=\frac{k-i}{n-t} > \frac{k-\p t}{n-t} = \frac{k(1-t/n)}{n-t} =
\frac{k(n-t)}{n(n-t)} = \frac{k}{n}=\p,
\]
and if $i > \p t$ then
\[
\q=\frac{k-i}{n-t} < \frac{k-\p t}{n-t} = \frac{k}{n}=\p.
\]
Both $P$ and $Q$ are unimodal and achieve their maximal value close to
$\p t$.

\begin{lemma}
  For sampling with replacement, define $\fmode{P} = (t+1)\frac{k}{n}-1$. If
  $i<\fmode{P}$ then $\prob{P}{t,i-1}<\prob{P}{t,i}$, if $i>\fmode{P}$
  then $\prob{P}{t,i+1}<\prob{P}{t,i}$.\newline
  For sampling with replacement, define $\fmode{Q} =(t+1)\frac{k+1}{n+2}-1$,
  If $i<\fmode{Q}$ then $\prob{Q}{t,i-1}<\prob{Q}{t,i}$, if $i>\fmode{Q}$
  then $\prob{Q}{t,i+1}<\prob{Q}{t,i}$.\newline
\end{lemma}
\begin{proof}
  The idea of the proof is to consider the ratio between the
  probabilities for consecutive values of $i$. Both distributions have
  closed form expressions:

  For sampling with replacement we have 
  \[
  \prob{P}{t,i}={t \choose i} \p^i (1-\p)^{t-i}
  \]
  From which we get that
  \[
  \frac{\prob{P}{t,i}}{\prob{P}{t,i}} = \frac{(t-i)\p}{(i+1)(1-\p)}
  \]
  Noting that the last expression is monotone decreasing with $i$ and
  solving for $\frac{\prob{P}{t,i}}{\prob{P}{t,i}}$ yields the first
  part of the lemma.\newline

  For sampling without replacement we have 
  \[
  \prob{Q}{t,i}=\frac{{k \choose i}{n-k \choose t-i}}{{n \choose t}}
  \]
  From which we get that
  \[
  \frac{\prob{Q}{t,i}}{\prob{Q}{t,i}} = \frac{(k-i)(t-i)}{(i+1)(n-k-t+i+1)}
  \]
  Again, the last expression is monotone decreasing with $i$ and
  solving for $\frac{\prob{Q}{t,i}}{\prob{Q}{t,i}}$ yields the second
  part of the lemma.
  \end{proof}


\iffalse
Let $S$ stand for $P$ or $Q$, The mode $S$ is a set of one or two
integers such that  $\mode{\prob{S}{t}} \doteq \argmax_{i} \prob{S}{t,i}$.
  
  For $0 \leq t \leq n$, for the distribution $\prob{S}{t}$ we denote the
  ``exact mode'' by a real valued number $0\leq \fmode{S} \leq t$,
  Which defines the mode as follows
  \begin{equation}\label{eqn:mode}
    \mode{\prob{S}{t}} = \begin{cases}
      0 & \mbox{if $k=0$} \\
    \lfloor \fmode{S}+1 \rfloor & \mbox{if $\fmode{S}$ is not
      an integer,} \\
      \fmode{S} \mbox{ and } \fmode{S}+1 & \mbox{if $\fmode{S}$ is an
        integer} \\
    t & \mbox{if $k=n$}
  \end{cases}
  \end{equation}
  
\fi

Next we give a recursive formulation for $P$ and $Q$. \newline
For $t=0$ we have that 
$\prob{P}{0,0}=\prob{Q}{0,0}=1$ and for all $i\neq 0$
$\prob{P}{0,i}=\prob{Q}{0,i}=0$.
For $t \geq 0$ the following recursion holds:
\[
\prob{P}{t+1,i}=\frac{k}{n} \prob{P}{t,i-1} + \left(1-\frac{k}{n}\right)\prob{P}{t,i}
\]
and
\[
\prob{Q}{t+1,i}=\frac{k-i+1}{n-t}\prob{Q}{t,i-1} + \left(1-\frac{k-i}{n-t}\right)\prob{Q}{t,i-1}
\]

For $0 \leq t \leq n$ and $0 \leq i \leq t$ we define the left and
right tails of the $P$ and $Q$ distributions as follows:
\[
\tail{L}{P}{t,i} \doteq \sum_{j=0}^i \prob{P}{t,j},\;\;\;\;\;\;
\tail{L}{Q}{t,i} \doteq \sum_{j=0}^i \prob{Q}{t,j}
\]
\[
\tail{R}{P}{t,i} \doteq \sum_{j=i}^t \prob{P}{t,j},\;\;\;\;\;\;
\tail{R}{Q}{t,i} \doteq \sum_{j=i}^t \prob{Q}{t,j}
\]
We can now state our main lemma regarding the tails of the
distribution $Q$.
\begin{lemma}
  For all $0\leq t \leq n$, let $\fmode{Q} = (t+1)\frac{k+1}{n+2}-1$,
  then 
  \begin{itemize}
    \item If $i < \lfloor \fmode{Q} \rfloor$ then $\tail{L}{Q}{t,i} < \tail{L}{P}{t,i}$
    \item If $i > \lceil \fmode{Q} \rceil$ then $\tail{R}{Q}{t,i} < \tail{R}{P}{t,i}$
    \end{itemize}
\end{lemma}

\subsection{End of section on sampling withuout replacement}
