\documentclass{article}
\usepackage[utf8]{inputenc}

\usepackage{amsthm}
\usepackage{amssymb}
\usepackage{amsmath}
\usepackage{color}

\usepackage{hyperref}
\usepackage{url}
\usepackage{times}
\usepackage[algo2e]{algorithm2e}

%\usepackage{fullpage}
%\usepackage{amsmath,amsfonts,amsthm,amssymb}
\usepackage{bbm}
\usepackage{graphics, graphicx, xcolor}
\usepackage{enumitem}
%\usepackage{verbatim}		% for misc commenting, etc.
\usepackage{stmaryrd}
\usepackage{float}
\usepackage[mathscr]{euscript}


\usepackage{geometry}
%% \geometry{a4paper,
%%   total={170mm,220mm},
%%   marginparwidth=80mm,
%% left=5mm,
%% right=85mm,
%% top=20mm,
%% }

% For algorithms
\usepackage{algorithm}
\usepackage{algorithmic}


\title{Non-Parametric Active Learning}
\author{Akshay Balsubramani, Yoav Freund, Shay Moran}

\newtheorem{theorem}{Theorem}[section]
\newtheorem{claim}{Claim}[section]
\newtheorem{corollary}{Corollary}[theorem]
\newtheorem{lemma}[theorem]{Lemma}
\newtheorem{assumption}[theorem]{assumption}
\newtheorem{definition}[theorem]{Definition}

\newtheorem{thm}{Theorem}%[section]
\newtheorem{lem}[thm]{Lemma}
\newtheorem{prop}[thm]{Proposition}
\newtheorem{cor}[thm]{Corollary}
\newtheorem{conj}[thm]{Conjecture}
\newtheorem{obs}[thm]{Observation}
\newtheorem{defn}[thm]{Definition}
\newtheorem{alg}{Algorithm}
\newtheorem{ass}{Assumption}
\newtheorem{examp}{Example}
\newtheorem{property}{Property}
\setcounter{MaxMatrixCols}{20}

\DeclareMathOperator{\id}{id}
\DeclareMathOperator{\tr}{tr}
\DeclareMathOperator*{\argmin}{arg\,min}
\DeclareMathOperator*{\argmax}{arg\,max}
\DeclareMathOperator{\sgn}{sgn}
\DeclareMathOperator{\Prtxt}{Pr}
\DeclareMathOperator{\var}{var}
\DeclareMathOperator{\poly}{poly}
\DeclareMathOperator{\polylog}{polylog}

\newcommand{\err}{\mbox{err}}
\newcommand{\X}{{\cal X}}
\newcommand{\Y}{{\cal Y}}
\newcommand{\D}{{\cal D}}
\newcommand{\B}{{\cal B}}
\newcommand{\x}{\vec{x}}
\newcommand{\y}{\vec{y}}
\newcommand{\vv}{\vec{v}}
\newcommand{\cc}{\vec{c}}

\newcommand{\K}{{\cal K}}
\newcommand{\restrictedto}{\triangleright}
\renewcommand{\SS}{{\cal S}} % Specialists
\newcommand{\CC}{{\cal C}}  % constraints

\newcommand{\outcome}{z}
\newcommand{\empoutcome}{\hat{\outcome}}
\newcommand{\polarity}{p}

\newcommand{\bd}[1]{\mathbf{#1}}  % for bolding symbols
\newcommand{\RR}{\mathbb{R}}      % Real numbers
\newcommand{\ZZ}{\mathbb{Z}}      % Integers
\newcommand{\NN}{\mathbb{N}}      % natural numbers
\newcommand{\RP}{\mathbb{RP}}      % real projective space
\newcommand{\Sp}{\mathbb{S}}
\newcommand{\HH}{\mathbb{H}}
\newcommand{\col}[1]{\left[\begin{matrix} #1 \end{matrix} \right]}
\newcommand{\comb}[2]{\binom{#1^2 + #2^2}{#1+#2}}
\newcommand{\vnorm}[1]{\left\lVert#1\right\rVert} % vector norm
\newcommand{\bfloor}[1]{\left\lfloor#1\right\rfloor} % floor function
\newcommand{\bceil}[1]{\left\lceil#1\right\rceil} % ceiling function
\newcommand{\ifn}{\mathbf{1}} % indicator function for sets
\newcommand{\EV}{\mathbb{E}} % expected value operator
\newcommand{\evp}[2]{\mathbb{E}_{#2} \left[#1\right]} % expected value operator
\newcommand{\abs}[1]{\left| #1 \right|}
\newcommand{\pr}[1]{\Prtxt \left(#1\right)}
\newcommand{\prp}[2]{\Prtxt_{#2} \left(#1\right)}
\newcommand{\ip}[2]{\left\langle #1, #2 \right\rangle}
\newcommand{\emperr}[2]{\widehat{\mbox{err}}_{#2} \left(#1\right)}
\newcommand{\Ex}{\mathbb{E}} % expected value operator


\newcommand{\pdis}[1]{P_{dis}\left(#1\right)}
\newcommand{\lrp}[1]{\left(#1\right)}
\newcommand{\lrb}[1]{\left[#1\right]}
\newcommand{\lrsetb}[1]{\left\{#1\right\}}

\newcommand{\corr}{\mbox{corr}}
\newcommand{\ones}{\mathbbm{1}}
\newcommand{\vA}{\mathbf{A}}
\newcommand{\va}{\mathbf{a}}
\newcommand{\vd}{\mathbf{d}} 
\newcommand{\vf}{\mathbf{f}}
\newcommand{\vF}{\mathbf{F}} 
\newcommand{\vI}{\mathbf{I}}  
\newcommand{\vh}{\mathbf{h}}
\newcommand{\vx}{\mathbf{x}}
\newcommand{\vb}{\mathbf{b}} 
\newcommand{\vu}{\mathbf{u}}   
\newcommand{\vl}{\mathbf{l}}
\newcommand{\vm}{\mathbf{m}}    
\newcommand{\vg}{\mathbf{g}}   
\newcommand{\vp}{\mathbf{p}}
\newcommand{\vq}{\mathbf{q}}
\newcommand{\vr}{\mathbf{r}}
\newcommand{\vs}{\mathbf{s}}
\newcommand{\vt}{\mathbf{t}}
\newcommand{\vw}{\mathbf{w}}
\newcommand{\vz}{\mathbf{z}}
\newcommand{\valpha}{\vec{\alpha}}
\newcommand{\vbeta}{\vec{\beta}}
\newcommand{\vzero}{\mathbf{0}}
\newcommand{\vone}{\mathbf{1}}

\newcommand{\cA}{\mathcal{A}}
\newcommand{\cB}{\mathcal{B}}
\newcommand{\cC}{\mathcal{C}}
\newcommand{\cH}{\mathcal{H}}
\newcommand{\cX}{\mathcal{X}}
\newcommand{\cY}{\mathcal{Y}}
\newcommand{\cZ}{\mathcal{Z}}
\newcommand{\cG}{\mathcal{G}}
\newcommand{\cD}{\mathcal{D}}
\newcommand{\cU}{\mathcal{U}}
\newcommand{\cS}{\mathcal{S}}
\newcommand{\cL}{\mathcal{L}}
\newcommand{\cN}{\mathcal{N}}
\newcommand{\cM}{\mathcal{M}}
\newcommand{\cF}{\mathcal{F}}
\newcommand{\cW}{\mathcal{W}}
\newcommand{\cE}{\mathcal{E}}
\newcommand{\cO}{\mathcal{O}}

\newcommand{\bias}{\text{bias}}
\newcommand{\ebias}{\widehat{\text{bias}}}

\newcommand{\sign}{\text{sign}}
\newcommand{\new}[1]{\textcolor{red}{#1}}

\newcommand{\I}{\mathcal{I}}
\newcommand{\N}{\mathbb{N}}
\newcommand{\ActiveLearn}{$\mathbf{ConfLearn}$}
\newcommand{\universe}[1]{{\cal #1}}

\newcommand{\samp}{S}
\newcommand{\usamp}{\underline{S}}

\newcommand{\comment}[3]{\marginpar{\textcolor{#2}{#1: #3}}}
%\newcommand{\comment}[3]{}
\newcommand{\shay}[1]{\comment{Shay}{red}{#1}}
\newcommand{\yoav}[1]{\comment{Yoav}{blue}{#1}}
\newcommand{\akshay}[1]{\comment{Akshay}{orange}{#1}}


\title{Non-Parametric Active Learning}
\author{Akshay Balsubramani, Yoav Freund, Shay Moran}
\begin{document}

\maketitle
\section{Introduction}
\iffalse
\shay{Was the Muffler ever used in the literature of Active Learning? }
In this paper we suggest a different approach to the problem of active
learning using balls. \new{Instead of using the min/max solution given by
Muffler}, we consider an Occam's razor approach by which we make the
simplest prediction possible given the constraints.
\fi

\section{Learning using Specialists}

We start with some standard definitions. A classifier is a
mapping from a domain $\cX$ to a binary label. For the sake of symmetry,
we define the binary labels to be $\{-1,+1\}$. 
We assume an (unknown) distribution $\D$ over $X \times \{-1,+1\}$.

In confidence rated classification (CRC) we allow the classifier to
output one of {\em three} labels: $\{-1,0,+1\}$, where the meaning of
$0$ is ``no prediction''. The performance of CRC is measured by two
parameters, one is the conditional accuracy $\D\bigl(c(x) \neq y | c(x) \neq
0\bigr)$, the other is the coverage, i.e. $\D\bigl(c(x) \neq 0\bigr)$.

We address a special case of CRC with {\em specialist} ensembles, 
in which the outputs of the classifier are either in $\{0,+1\}$ (positive specialists) or in
$\{0,-1\}$ (negative specialists). 
The support of a specialist is a set $A$ on which the classification of
the specialist is non-zero. Given such a set, we denote by $A^+: \cX \to \{0,+1\}$ the
function the corresponds to the positive specialist with support $A$, and by $A^-$ the
function that corresponds to the negative specialist.


\new{$\ActiveLearn$ is a transductive algorithm}.
\shay{What does this mean? How is it different from
  standard algorithms in pool-based active learning?}
\yoav{The reason that transductive is easier is that the set of
  specialists is finite and we don't need to worry about intersections
  and unions of infinite sets when defining the boundary and the
  interior. The difference with pools of generalists comes from the
  fact that that specialists can concentrate on smaller parts of the
  input space.}
 In other words, the input to the algorithm is a set of unlabeled instances $\universe{X} =
[x_1,\ldots,x_N]$. All probabilities and expectations are defined with
respect to the uniform distribution over $\universe{X}$. The hidden
target is represented by a conditional distribution over the binary label
$y \in \{-1,+1\}$. We define the {\em bias} of instance $x$ as
\[
\bias(x) \doteq P(y=+1|X=x) - P(y=-1|X=x)
\]
Let $B$ be (the support set of) a specialist. 
\new{The {\em true size} of $B$ is defined by
\[\lvert B \rvert =  {\sum_{x \in \universe{X}} \ones(x \in B)}.\]}
The {\em true bias} of the specialist as:
\[
\bias(B) \doteq
\frac{\sum_{x \in \universe{X}} \bias(x) \ones(x \in B)}
{\lvert B \rvert}
\new{= P(y=+1 \vert X\in B) - P(y=-1 \vert X\in B)}.
\]

\ActiveLearn\ can query the label for any instance $x \in \universe{X}$, the
response to the query is a label drawn independently at random
according to $P(y=+1 | X=x)$.\footnote{If the same instance is queried
more than once, the responses are independent and can be different.}

Let $\samp= \left\langle (\x_1,y_1),\ldots,(\x_m,y_m) \right\rangle$ be a sequence
of queried instances and the corresponding labels. 
\new{The {\em empirical size} of $B$ with respect to $\samp$ is defined by
\[\lvert B \rvert_S =  {\sum_{(x,y) \in \samp} \ones(x \in B)}.\]}
\shay{The definition of $k(B)$ was missing. 
I added it and changed the notation from $k(B)$ to $\lvert B\rvert_S$.
If you agree then we should also change it in the simultaneous convergence proof.}
The {\em empirical bias} of a specialist $B$ with respect to $\samp$ to be 
\[
\ebias(B,\samp) \doteq
\frac{\sum_{(x,y) \in \samp} y \ones(x \in B)}{\lvert B\rvert_S}.
\]

\iffalse
When learning using standard classifiers (to which we refer here as
{\em generalists} the learning algorithm typically searches for the
``empirical risk minimizer'' or ERM. The role of generalization bounds
is to show that the generalization error of the ERM is not much larger
than the training error. In contrast, when learning using specialists,
the rule that is output is not the ERM, instead, it is a rule that is
a combination of many specialists. The specialists will typically have
small (low probability) support. The combined classifier will have a
higher probability of a non-zero prediction, but it might still
abstain on a significant part of the space where the conditional
probability of a $+1$ label is close to $1/2$. In this work 	
we calculate convergence bounds {\em inside the
  learning algorithm}, this is similar to work on ``self-bounding
algorithms''.

The algorithm uses the bounds to compute confidence
intervals on $\bias(B)$ the probability of the label $+1$
conditioned on the instance $x$ being in the support set $B$.
\fi

Let $\cB$ be a collection of support sets corresponding to a set of

specialists. We assume that $\cB$ has VC dimension $d$. 

We now state a convergence bound that establishes the strong (almost sure) \emph{simultaneous} convergence 
of the empirical biases of a set of specialists to their true biases, 
at rates which depend on the number of samples collected (roughly the support size) for each specialist.
The proof is given in Section~(\ref{sec:convergence-bounds}).

%% \yoav{I don't think that the term ``uniform'' is appropriate here,
%%   because the bound is not uniform - there is a different bound for
%%   each rule, depending on $k$.}

%% \akshay{Changed it to say the plain mathematical statement of strong
%%   simultaneous convergence. Also simplified some notation, mostly
%%   around "conforming sets," and omitting the space $\cX$ as an
%%   explicit argument.}

\begin{theorem}\label{thm:UCECM}
Let $\cB$ be a set of specialists with VC dimension $d$, let
$\delta>0$ ,and let $\samp$ be a labeled sample of size $n$.  The
following holds with probability at least $1-\delta$ over the random
choice of $\samp$:

\[\forall B\in\cB :\;\;
\lvert \bias(B) - \ebias(B,\samp) \bigr) \rvert
\leq \sqrt{\frac{k_0}{\new{\lvert B\rvert_S}}},
\]
where $k_0 =200\bigl(d\ln(3n) + \ln(1/\delta)\bigr)$.
\end{theorem}


\newcommand{\set}{G}
\newcommand{\sample}[1]{{\mathbf #1}}
\newcommand{\constr}[3]{#1 \left(#2,\sample{#3}\right)}
  
Using this theorem and given a sample $\sample{S}$, 
\ActiveLearn\ identifies those specialists $B \in \cB$ for which the sign of
$\bias(B)$ can be determined (with high probability). 
\begin{itemize}
\item The \emph{probably positive specialists} $\cB^{+} (\sample{S})$ contains all $B \in \cB$ such that
  $\ebias(B) \new{>} \sqrt{\frac{k_0}{\new{\lvert B\rvert_S}}}$.
\item The \emph{probably negative specialists} $\cB^{-} (\sample{S})$ contains all $B \in \cB$ such that
  $\ebias(B) \new{<} -\sqrt{\frac{k_0}{\new{\lvert B\rvert_S}}}$.
\item The \emph{undetermined specialists} $\cB^{0} (\sample{S})$ contains $B \in \cB$ that are neither in
  $\cB^{+} (\sample{S})$ nor in $\cB^{-} (\sample{S})$.
\end{itemize}

Note that the statement of the theorem holds simultaneously for all $B \in
\cB$. Therefor with probability $1-\delta$ we have that $\forall B \in
\cB^{+} (\sample{S})$, $\bias(B)>0$ and $\forall B \in \cB^{-}
(\sample{S})$, $\bias(B)<0$. As \ActiveLearn\ accumulates labeled
examples, it also accumulates positive and negative specialists. We
denote the union of the positive specialists and the union of the negative
specialists by $\cB^+$ and $\cB^-$ respectively.


\section{The active learning algorithm}

\newcommand{\conform}{{\cal C}}
\newcommand{\nonconform}{{\cal N}}
\newcommand{\cover}{{\cal M}}
Like many other active learning algorithms~\cite{}, the intuition
behind \ActiveLearn is to preferably query instances that are close
to the decision boundary. A related viewpoint is to prefer instances
on which there is disagreement among the best rules.

\ActiveLearn\ operates in epochs. In each epoch it uses the set of
determined specialists accumulated in previous epochs:
$\cB^+_t,\cB^-_t$ to partition $\universe{X}$ into {\em interior} and
{\em boundary} sets, as in \cite{ChadhuriDasgupta2015}. 

We define a partition of a set of points $\universe{X}$ into effective
interior and boundary sets. 
%Given a sample $S$, let $\cB^{\pm} := \cB (\pm, S)$ be positive and negative conforming sets on $S$. 

\newcommand{\algInterior}{{\cal C}}
\newcommand{\algBoundary}{{\cal N}}
\newcommand{\algCover}{{\cal M}}

\begin{itemize}
\item {\em  Effective Interior Sets:} There are two interior sets, one for each label.
  Let $y \in \{-1,+1\}$ be that label. We define the $y$-interior subset of $\X$ as
  \[
  \algInterior^{y} (\universe{X},\cB^{\pm 1}) =
  \left\{ x \in \universe{X} \left|
  \begin{array}{c}
    \exists B \in \cB^{y} \mbox{ s.t. } x \in B
    \\ \mbox{ and } \\
    \forall A \in \cB^{-y} \mbox{ s.t. } x \in
    A,\;\;\;
    \exists D \in \cB^{y} \mbox{ s.t. } D \in C
    \mbox{ and } D \subset A
  \end{array}
  \right. 
  \right\}.
\]
\new{Thus, $x$ is in the $y$-interior if all the minimal specialists in $\cB^y\cup\cB^{-y}$
  that contain $x$ are in $\cB^{y}$.}

\item {\em Effective Boundary:} This set contains all points that are not in either interior set:
  \[
  \algBoundary(\universe{X},\cB^{\pm 1}) =
  \universe{X} \setminus \lrb{ \algInterior^{+1} (\universe{X},\cB^{\pm 1}) \cup \algInterior^{-1} (\universe{X},\cB^{\pm 1}) }.
  \]
  \new{Thus, $x$ is in the effective boundary if there are are two minimal specialists in $\cB^{+1}\cup\cB^{-1}$ that contain $x$ and one of them is in $\cB^{+1}$ and the other is in $\cB^{-1}$.}

\item {\em Boundary Cover:} To ensure that the specialists that
  contain boundary points recieve a sufficient number of queries we
  need to expand the boundary to a set we call the boundary cover.
  \[
  \algCover(p,\X,\cB^{\pm 1}) =
  \bigcup \left\{ B \in \cB^{\pm 1} \left|
  \D(B)<p,\;\mbox { and } B \cap \algBoundary(\universe{X},\cB^{\pm 1}) \neq \emptyset
  \right.
  \right\}.
  \]
\item {\em Boundary cover:} This set contains all of the specialist of
  probability $\leq p$ that contain a point of the effective boundary.
  \[
  \cover(p,\cB^{\pm 1}) =
  \left\{ x \in \X \left|
  \exists B \in \cB \mbox{ s.t. } \D(B) \leq p \mbox{ and } B \cap
  \nonconform(\cB^{\pm 1}) \neq \empty
  \right.\right\}
  \]
\end{itemize}


\subsection{Algorithm}

With these definitions, we can now describe the active learning algorithm \ActiveLearn.

\newcommand{\boundaryIter}{\universe{U}}
\newcommand{\all}{\universe{X}}
\noindent
\begin{enumerate}
\item Define the initial boundary cover to be all of the instances
  $\boundaryIter_0=\all$
\item Query \new{$n_0$} points uniformly at random.
  \shay{How is $n_0$ defined?}
  \yoav{It is a parameter that will come in the analysis. We could set
    $n_0=1$. The algorithm is equivalent to the Passive algorithm
    until the boundary is different from the whole set $\X$}
\item for $t=1,2,\ldots$:
\begin{enumerate}
\item Let $p_t = 2^{-t}$
\item Let $m_t$ be the number of points in the boundary cover.
  Query $4m_t$ points chosen uniformly at random from $\all$ and $4m_t$
  points chosen uniformly at random from $\boundaryIter_{t-1}$. Define
  $\sample{S}_t$ and $\sample{U}_t$ to be the corresponding labeled
  datasets.
\item Partition the specialists in $\cB$ into conforming and non-conforming:
  $$
  \mbox{For } s \in \{-1,0,1\}:\;\; \cB^s_t := \cB^{s} \lrp{\bigcup_{i=1}^t \new{\sample{S}_i}}
  $$
  \shay{Should it be $\cB^{s} \lrp{\bigcup_{i=1}^t
      {\sample{S}_i\cup\sample{U}_i}}$?}
  \yoav{No, The boundary sets are not guaranteed to always
    shrink. There might be an ``unknown unknown'' region which creates
    a new boundary set. In fact, I think that the analysis will only
    depend on the the size of $\sample{S}_i$.}
  \item Define the {\em constrained set of specialists}:
  $
  \cC_t := \left\{ B \cap \boundaryIter_{t-1} \left| B \in \cB \right.\right\}
  $
\item Partition $\cC_t$ into conforming and non-conforming sets:
$$  \mbox{For } s \in \{-1,0,1\}:\;\; 
  \cC^s_t = \cC^{s} \lrp{\bigcup_{i=1}^t \sample{S}_i \cup \sample{U}_t}$$
\item Combine the conforming specialists:
  $
  \mbox{For } s \in \{-1,0,1\}:\;\; \cA^s_t = \cB^s_t \cup \cC^s_t
  $
  \shay{Should $\cA^s_t$ contain $\cA^s_{t-1}$?
Currently it does not contain constrained specialists from the previous iterations.}
\item Define a new effective boundary set:
  $
  \boundaryIter_t = \algCover(2^{-t},\X,\cA^{\pm 1})
  $
\end{enumerate}
\end{enumerate}
\yoav{The set of samples that can be used to evaluate specialists in
  $\cC_t$ is potentially larger. When there is no change of mind the
  unconforming sets shrinks with each epoch. The general rule is: you
  can use any samples from a batch whose support set contains the specialist.}


\section{Analysis}

%\new{Shay: What is the main statement we wish to prove in the Analysis section?}

In this section we give sufficient conditions under which
the number of queries of $\ActiveLearn$ is significantly smaller (in a
sense to be defined) than the number of queries required when using
passive sampling.

The analysis is inspired by the convergence analysis of kNN done
in~\cite{ChadhuriDasgupta2015}.  The idea we use here is to
characterize the true conditional distribution (\new{defined below}). This characterization
will allow us to bound the number of queries made by $\ActiveLearn$.

Let $\D$ define the joint distribution over $X \times \{-1,+1\}$. The
transductive sample $\X$ is chosen IID according to the marginal
$\D|_{X}$. The answer to a query $x \in \X$ is chosen according to the
conditional distribution $\D(y,x)$.
%\shay{``True distribution'' is used in a confusing way. 
%Perhaps our main theorems should consider the standard setting of an unknown 
%(not necessarily uniform)  distribution over the whole domain, and in our proofs we can use a reduction
%to the transductive setting.}

\newcommand{\restricted}[2]{{#1}(#2)}

We denote by $\cB_x$ the set of specialists $B \in \cB$ that {\em
  cover} $x$, i.e. $x \in \cB$.

%We define the {\em neighborhood} of a
%point $x$ to be the union of the sets in $N_x=\cup{B | B \in \cB_x}$ 

Call a specialist $B$ uniformly $(s,\Delta)$-biased if $s\cdot\bias(x) \geq \Delta$ for all $x \in B$.
Define restricted sets of specialists to be subsets of $\cB_x$ that
satisfy the following conditions on $\bias(B)$ and $\D(B)$:
\[ \restricted{\cB_x}{s,\Delta,\leq p}
=
\left\{
B \in \cB_{x} \left| s \cdot\bias(B) \geq \Delta \mbox{ and } \D(B) \leq p
\right.\right\}
\]
Instead of $\leq p$, the condition in this definition can be
$<p$ or $>p$ or $\geq p$.

We now define the $s,p_1,p_2,\Delta$ interior points.

\newcommand{\interior}[3]{{#1}^{#2}_{#3}}
\newcommand{\boundry}[1]{\cG_{#1}}
\begin{definition}
For $s = \pm 1$, $p_1>p_2>0$, $\Delta>0$ the {\em interior set}
$\interior{\all}{s}{p_1,p_2,\Delta}$ contains all points $x \in \all$ such that
\begin{enumerate}
\item {\bf No mind changes:} $\forall B \in
  \restricted{\cB_x}{s,\Delta,\leq p_1}$, $B$ is \new{uniformly} $(s,\Delta)$-biased.
  \new{Shay: What does ``uniformly'' means?}
\item {\bf Witness:} $\exists B \in \restricted{\cB_x}{s,\Delta,\leq p_1}$ such that $\D(B) \geq p_2$.
\item {\bf Masking:} For all $C \in \restricted{\cB_x}{-s,0,> p_1}$
  there exists $B \in \restricted{\cB_x}{s,\Delta,\geq p_2}$ such
  that $B \subseteq C$.
\end{enumerate}
The {\em boundary set} is the complement of the two interior sets:
$$\boundry{p_1,p_2,\Delta} = \all -
(\interior{\all}{+1}{p_1,p_2,\Delta} \cup
\interior{\all}{-1}{p_1,p_2,\Delta}) $$
\end{definition}

Some intuition behind this definition. The ``no mind change''
condition guarantees that the prediction using specialists whose
weight is large will not be contradicted by lighter
specialists.~\footnote{Note that this is a condition defined for the
  analysis. The algorithm itsef can never be sure that there will not
  be a mind-change.} The ``Witness'' condition
guarantees that there is at least one specialist with weight
larger than $p_2$ predicts in the right direction. The third condition
``Masking'' is related to the masking condition used in \ActiveLearn.

The boundary sets are nested. If $p_1\geq p'_1$, $p_2 \leq p'_2$ and $\Delta \geq \Delta'$, then
$\boundry{p_1,p_2,\Delta} \subseteq
\boundry{p'_1,p'_2,\Delta'}$.

The Non-conforming sets of \ActiveLearn\ are related to the boundary
sets. It is therefore natural to characterize the query complexity of
a distribution using the dependence of the probability of the boundary
set on $p_1,p_2$ and $\Delta$.

\begin{definition}
  For a given distribution $\D$ and specialist class $\cB$, we define
  the $(p_1,p_2,\Delta)$-boundary size to be
  $$V(p_1,p_2,\Delta) =
  \D(\boundry{p_1,p_2,\Delta})$$
\end{definition}

As $\ActiveLearn$\ is at its core a nearest neighbor algorithm, we
quantify its performance by considering the difference between its error
rate and the error rate of the Bayes optimal
algorithm. We say that the algorithm is $\epsilon$-competitive if its
error rate is at most $\epsilon$ higher than the error of the Bayes algorithm.

\begin{theorem}
Let $a,b,c,d>0$ be positive constants. If $V(a/n,b/n,c/\sqrt{n}) <
d/n$ then after querying $\log(1/\epsilon)$ points the prediction of
the algorithm is $\epsilon$-competitive.
\end{theorem}
\new{Shay: Something seems off with this theorem; it gives the same bound on the query complexity for any set of specialists! This is clearly false: for example, imagine there is only one singleton specialist $\{x\}$.
Clearly, in this case the algorithm will not output a classifier that is competitive with the Bayes optimal.}

\section{\color{red} [old material] Ball Specialists}

We restrict our attention to a special case which corresponds,
roughly, to nearest neighbor methods.
\begin{enumerate}
\item The input space $\X$ is a finite set in $R^d$. We assume a
  uniform distribution over $\X$.
  \item We consider labels $y\in \{-1,+1\}$. For There is a fixed but
    unknown conditional probability defined over $\X$, i.e. $P(Y=+1 |
    X=\x)$. Our goal is to estimate whether $P(Y=+1|X=\x)$ is smaller
    or larger than $1/2$.
  \item
    The notation is simpler if we use expected values instead of
    probabilities. We use the term {\em bias} of a ball to refer to the conditional
    expectation of the label for a ball by
    \[
    \bias(\x) \doteq P(Y=+1|X=\x) - P(Y=-1|X=\x)
    \]
  \item
    The rules that we use are ``specialists'' that are balls. The set
    $\B$ contains all rules of the form
    \[
    B_{r,\cc,s}(\x) =
    \begin{cases}
      s & \text{if } \| \cc- \x \| \leq r \\
    0 & \text{otherwise }
    \end{cases}
    \]
    Where $r \geq 0$ is the radius of the ball, $\cc \in R^d$ is the
    center of the ball and $s \in \{-1,+1\}$ is the polarity of the ball.
    We will drop the subscripts of $B$ when clear from context.
  \item
    We use $\x \in B$ to indicate that $B(\x) \neq 0$.
  \item
    We denote the {\em probability} of a ball $B$ by $p(B) \doteq
    \frac{|B|}{|\X|}$.
  \item
    We define term {\em bias} of a ball to refer to the conditional
    expectation of the label for a ball by
    \[
    \bias(B) \doteq E\left( y|\x \in B \right).
    \]
  \item We define a sample $\samp$ as a sequence of labeled examples:
    \[\samp= \left\langle (\x_1,y_1),(\x_2,y_2),\ldots,(\x_m,y_m)
    \right\rangle \]
    Where $\x_i$ are chosen uniformly at random (with replacement)
    from $\X$ and $y_i$ are chosen according to the (unknown)
    conditional distribution of the label given $\x$.
  \item Given a sample $\samp$, we define the number of instances in
    $(\x,y) \in \samp$ such that $\x \in B$ to be the {\em size} of
    $B$ according to the sample $\samp$ and denote it by $k_{\samp}$. We define
    the {\em empirical probability} of the ball $B$ according to
    $\samp$ by $p_{\samp}(B) \doteq k_{\samp}(B)/|\samp|$.
  \item Given a sample $\samp$ we define the estimate of the
    $\bias(B)$ to be
    \[
    \ebias_{\samp}(B) = \frac{\sum_{i=1}^m y_i B(\x_i)}{\sum_{i=1}^m B(\x_i)}
    \]
  \item Using uniform convergence bounds we define for each ball $B$
    a confidence interval:
    $[l,h]=[\ebias(B)-\Delta,\ebias(B)+\Delta]$.
    if $l>0$ we say that $B$ imposes a {\em positive constraint}, if
    $h<0$ we say that $B$ imposes a {\em negative constraint}, i
    $l\leq 0 \leq h$ we say that $B$ does not impose a constraint.
\end{enumerate}

\section{Universal convergence bounds for specialists}
\label{sec:convergence-bounds}

\yoav{I think that stating using the general term $A$ is an event,
  rather than the more problem specific $y=+1$ is confusing. I like
  the discussion but I would like to shorten it and not use ``too good
  to be true'' non-theorems.}
 \shay{ I believe this statement can be interesting beyond this context (e.g. it is related to discrepancy). Therefore,  I think it is better if one can read and understand the statement without knowing our definitions of bias, etc\ldots
We can explain how we use it in detail to avoid confusion.
I can remove the ``too good to be true'' part after I am back to Princeton.}



Let $p$ be a distribution over $X$.
Let $\cB$ be a collection of events, and let $A$ be an event.
Assume that $\cB$ has a finite VC dimension, denoted by $d$.
Consider $n$ independent samples from~$p$, denoted by $x_1,\ldots,x_n$.
We would like to estimate $p(A \vert B)$ simultaneously for all $B\in \B$.
It is natural to consider the empirical estimates:
\[p_n(A\vert B)=\frac{\sum_i 1_{[x_i\in A \cap B]}}{\sum_i 1_{[x_i\in B]}}.\]
We study how well and when these estimates approximate the underlying ones simultaneously.

To demonstrate the kind of statements we would like to derive,
consider the case where there is only one event $B$ in $\cB$, 
and let $k(B)=\sum_i 1_{[x_i\in B]}$.
A Chernoff bound implies that conditioned on the event that~$k(B)>0$, 
it holds with probability at least $1-\delta$ that:
\begin{equation}\label{eq:chernoff}
\bigl\lvert p(A\vert B) - p_n(A \vert B) \bigr\rvert \leq \sqrt{\frac{2\ln(1/\delta)}{k(B)}}.
\end{equation}
To derive it use that conditioned on $x_i\in B$, the event $x_i\in A$ has probability $p(A\vert B)$, 
and therefore the random variable ``$k(B)\cdot p_n(A \vert B)$'' has distribution $\mbox{Bin}(k(B), p(A\vert B))$.

Note that the bound on the probability in \ref{eq:chernoff} depends on $k(B)$
and therefore is random.
We stress that this is the type of statement we want:
the more samples belong to $B$ | the more certain we are with the empirical estimate.

We would like to prove something like this:
\begin{theorem}[A too good to be true UCECM]\label{thm:toogood}
The following event occurs with probability at least $1-\delta$:
\[\bigl(\forall B\in\B\bigr):\bigl\lvert p(A \vert B) - p_n(A \vert B) \bigr\rvert \leq O\Bigl(\sqrt{\frac{d\ln(1/\delta)}{k(B)}}\Bigr),\]
where $k(B) = \sum_{i=1}^n 1[x_i\in B]$.
\end{theorem}
\ref{thm:toogood} is, unfortunately, false. 
As an example, consider the probability space defined by drawing uniformly $x\sim[n]$,
and then coloring $x$ by $c_x\in\{\pm 1\}$ uniformly.
For each $i$ let $B_i$ denote the event that $i$ was drawn,
and let $A$ denote the event that the drawn color was  $+1$.
(formally, $B_i = \{i\}\times\{\pm 1\}$, and $A=[n]\times\{+1\}$).
One can verify that the VC dimension of $\B=\{B_i : i\leq n\}$ is $1$.

\ref{thm:toogood} fails in this setting:
indeed, if we sample $n$ times from this space 
then with a constant probability there will be some  $j$
that is assigned with the same color in each sample,  say $+1$, 
and will be picked some $\Theta(\ln n/\ln\ln n)$ times\footnote{{This follows from analyzing the maximal bin
in a uniform assignment of $\Theta(n)$ balls into $n$ bins~\cite{bins}}}.
Therefore, $p_n(A\vert B_i) = 1$, $p(A\vert B_i)=1/2$,
and $1-(1/2) \gg \sqrt{\ln\ln n/\ln n}$.

We prove the following (slightly weaker) variant:
\begin{theorem}[UCECM]\label{thm:UCECM}
The following event occurs with probability at least $1-\delta$:
\[\bigl(\forall B\in\B\bigr):\bigl\lvert p(A \vert B) - p_n(A \vert B) \bigr\rvert \leq 
\sqrt{\frac{k_0}{k(B)}},\]
where $k_0 =200\bigl(d\ln(3n) + \ln(1/\delta)\bigr)$, and $k(B) = \sum_{i=1}^n 1[x_i\in B]$.
\end{theorem}

\paragraph{Discussion.}
\shay{Analyze this comparison more clearly.
I hope we can derive a statement like:
our bound becomes meaningful
for $B$ such that $p(B)\geq d\ln n/n$
but the naive bound requires $p(B)\geq \sqrt{d/n}$.}

\ref{thm:UCECM} can be combined with the following uniform convergence 
variant to yield a bound on the minimal $n$ for which $p_n(A \vert B)$ 
is a non-trivial approximation of $p(A \vert B)$
(as explained below, this $n$ is some $\tilde O\bigl(1/p(B)\bigr)$).
\begin{lemma}\label{lem:uconeside}
Set $k_0$ like in \ref{thm:UCECM}. Then with probability at least $1-\delta$:
\[
(\forall B\in \B):~ k(B) \geq \frac{1}{4}np(B) - {k_0}.
\]
\end{lemma}
This implies that once $n$ large enough so that $p(B)=\tilde\Omega(k/n)=\tilde\Omega\bigl(\frac{d}{n}\bigr)$
then the empirical estimate $p_n(A\vert B)$ becomes non-trivial.
In the context of specialists, this means that the empirical
biases provide meaningful estimates of the true biases 
for specialists whose measure is $\tilde\Omega\bigl(\frac{d}{n}\bigr)$.
Note the resemblance with the learning rate in realizable settings.
%
%indeed, it implies that roughly $p(B)\pm \sqrt{d/n}$ 
%empirical points from the sample are will indeed belong to $B$
%(with high probability), and so, once $n$ is sufficiently large,
%one can replace the radius of the confidence interval by 
%$\Theta\Bigl(\sqrt{\frac{d\ln n}{p(B)n - \sqrt{dn}}}\Bigr)$, 
%which is for a large $n$ roughly $\sqrt{\frac{d\ln n}{p(B)n}}$.

A relevant remark is that a weaker statement than \ref{thm:UCECM}
can be derived as a corollary of the classical uniform convergence
result~\cite{vapnik}. 
Indeed, since the VC dimension of $\{B\cap A : i\in \I\}$ is at most $d$, it follows that 
\[p_n(A\vert B)\approx\frac{p(A\cap B) \pm \sqrt{d/n}}{p(B)\pm \sqrt{d/n}}.\]
However, this bound guarantees non-trivial estimates only once $p(B)$ is roughly $\sqrt{d/n}$.
This is similar to the learning rate in the agnostic learning.

Another practical advantage of the uniform convergence bound in \ref{thm:UCECM} is its dependence on $k(B)$,
which may be thought of as a ``lucky-case'' bound:
if (by ``luck'') many points from the sample belong to $B$ 
then $p_n(A\vert B)$ is a good estimate of $p(A\vert B)$.


\subsection{Specific conditions}

I believe either of the following conditions implies
$\epsilon$-approximation using balls.

\begin{enumerate}
\item
  {\bf The conditional probability is Lipshitz smooth}. In other words, for
any $\x,\y \in \X$:
\[
|\bias(\x) - \bias(\y)| \leq \alpha \|\x-\y\|_2^{\beta}
\]
\shay{This condition seems relevant only when the specialists ``behave like balls'' }
\item
The bias is bounded away from zero and {\bf the boundary is low
dimensional}. More technically:
\begin{itemize}
\item There exists some $\epsilon_0$ such that $\forall \x \in \X$, $|\bias(\x)|>\epsilon_0$.
\item Recall that $\X \subset R^d$. Define boundary balls as balls
  that contain and element with positive bias and an element with
  negative bias. Let $G_{\epsilon}$ be the union of all boundary balls
  with probability at most $\epsilon$. We say that the boundary has
  low dimension if $P_{\D}(G_{\epsilon}) \to 0$ as $\epsilon \to 0$
  (one has to be a bit careful in the definition here because $\X$ is finite, so
  $G_{1/2|\X|}=\emptyset$ trivially).  
  \shay{It seems these definitions make more sense for infinite domains.
  I suggest that we assume an infinite domain and reduce to the transductive finite setting
  in our proofs.}
\end{itemize}
\end{enumerate}

\appendix

\section{Proof of \ref{thm:UCECM}}

The idea is to follow the standard argument due to~\cite{vapnik} 
which derives the standard uniform convergence result (and to modify it accordingly). 
We follow the exposition of~\cite{anthony}:
consider a double-sample of size $2n$ from~$p$, denoted by $x_1,\ldots,x_{2n}$.
Let $S_1$ denote the first half of the sample and $S_2$ the second.
Define $E_1$ to be the event whose probability we want to bound:
\[E_1 = \Bigl\{ S_1\in X^n : (\exists B\in \B):~ 
\bigl\lvert p(A \vert B) - p_{n,1}(A \vert B) \bigr\rvert > 
\sqrt{\frac{k_0}{k_{1}(B)}} \Bigr\},\]
where $p_{n,1}$ is the empirical measure induced by $S_1$, 
and $k_{1}(B)=\sum_{x \in S_1} 1[x \in B]$, 
with $p_{n,2}, k_{2}$ being defined similarly.
Let $E_2$ denote the event
\[E_2 = 
\Bigl\{
S_1S_2\in X^{2n} : (\exists B\in\B):~
\bigl\lvert p_{n,1}(A \vert B)   -  p_{n,2}(A \vert B) \bigr\rvert >  
{\frac{1}{2}}\sqrt{\frac{k_0}{k(B)}}
\Bigr\},
\]
where $k(B) = k_{1}(B)+k_{2}(B)$.
The strategy of showing that $\Pr[E_1]$ is by reducing it to showing that $\Pr[E_2]$ 
is small, and then to show that the latter is small using a standard \emph{symmetrization} argument. 
For the first part, we would like to argue like in~\cite{anthony}, that
\begin{equation}\label{eq:anthony} 
(\forall S_1\in E_1): \Pr_{S_2}[S_1S_2\in E_2]\geq \frac{1}{100},
\end{equation}
which would imply that $\Pr[E_1]\leq 100\Pr[E_2]$ and yield the reduction.
However, \eqref{eq:anthony} does not necessarily hold: indeed consider
$S_1\in E_1$ such that there is a single $B$ for which 
\[
\bigl\lvert p(A \vert B) - p_{n,1}(A \vert B) \bigr\rvert > 
\sqrt{\frac{k_0}{k_{1}(B)}}
\]
and further assume that 
(i) $B$ has a tiny measure, say $p(B) = 1/n$,
(ii) $p(A\vert B)= 1/2$, and
(iii) $p_{n,1}(A\vert B) = 0$.
Therefore, $k_2(B)=1$ with probability at least $1/4$
and therefore $p_{n,2}(A \vert B)=p_{n,1}(A \vert B)$
with probability at least $1/8$, which is the negation of $E_2$.

To get around this, we introduce an auxiliary event $F$, defined by
\[F = \Bigl\{S_1\in X^{n} : (\forall B\in\B): k_{1}(B) \geq k_0 \implies p(B)\geq \frac{k_{1}(B)}{4n}\Bigr\}. \]
\yoav{I find it more natural to define the complement of $F$, i.e. the
  set of samples for which there exists a ball where the bad thing happens.}
Note that $F$ discards the problematic example from above 
(by constraining every $B$ that witnesses $E_1$ to have a large probability). 
As we will see, $F$ is a typical event that happens with high probability,
and so, it enables us to replace \ref{eq:anthony} with the following:
\begin{lemma}\label{lem:reduction}
If $\Pr[E_1]\geq \delta$ then 
\begin{enumerate}
\item $\Pr[E_1\cap F] \geq \frac{1}{2}\Pr[E_1]$, and
\item $(\forall S_1\in E_1\cap F): \Pr_{S_2}[S_1S_2\in E_2] \geq \frac{1}{8}$. 
\end{enumerate}
\end{lemma}
We defer the proof of \ref{lem:reduction} to a later section, 
and assume it for now towards proving \ref{thm:UCECM}.

\ref{lem:reduction} yields the following win-win situation:
either $\Pr[E_1 \leq \delta]$ and we are done, 
or that $\Pr[E_1]\leq 16\Pr[E_2]$:
\[\frac{\Pr[E_2]}{\Pr[E_1]} \geq  \frac{1}{2}\frac{\Pr[E_2]}{\Pr[E_1\cap F]} \geq \frac{\Pr[E_2 \vert E_1\cap F]}{2} \geq \frac{1}{16},\]
where the first inequality uses the first item of \ref{lem:reduction}, 
the second inequality follows by definition of conditional probability,
and the last inequality is implied by the the second item of \ref{lem:reduction}.


We proceed to the standard symmetrization argument
that establishes $\Pr[E_2]\leq\delta/16$. 
Instead of sampling $S_1S_2\sim p^{2n}$,
consider the following equivalent process:
\begin{itemize}
\item[(i)] Sample $S\sim p^{2n}$.
\item[(ii)] Partition $S$ uniformly into two subsamples $S_1,S_2$, each of size $n$.
\end{itemize}
The following lemma implies that $\Pr[E_2]\leq \delta/16$, and finishes the proof.
\begin{lemma}\label{lem:e2}
For every $S\in X^{2n}$
\[\Pr_{S\to S_1S_2}\bigl[S_1S_2\in E_2\bigr]\leq \frac{\delta}{16},\]
where the randomness is over the uniform partitioning
of $S$ into $S_1,S_2$.
\end{lemma}
\qed

\subsection{Proof of \ref{lem:reduction}}

\paragraph{Item {\it 1}.}
We begin with the first item;
it suffices to show that $\Pr[F]\geq 1-\delta/2$.
This follows from the standard uniform convergence bound with
the difference that of using the multiplicative Chernoff bound instead
of the additive bound:
Consider a double sample $S=(\samp_1,\samp_2)\sim p^{2n}$.
Let $F_1$ denote the event whose probability we want to bound:
\[
F_1 =\bigl\{ S_1\in X^n : (\exists {B\in\B}):
  k_{1}(B) \geq k_0 \mbox{ and } p(B) \leq \frac{k_{1}(B)}{4n}\bigr\}, 
\]
and let $F_2$ denote the event:
\[
F_2 = 
\bigl\{ S_1S_2\in X^{2n} : (\exists {B\in\B}):
  k_{1}(B) \geq k_0 \mbox{ and } k_{2}(B) \leq \frac{k_{1}(B)}{2}\bigr\}.
\]
The proof follows from the following two lemmas:
\begin{lemma}\label{lem:aux11}
$\Pr[F_1]\leq 10\Pr[F_2].$
\end{lemma}
\begin{lemma}\label{lem:aux12}
$\Pr[F_2]\leq \delta/20.$
\end{lemma}
\begin{proof}[Proof of Lemma~\ref{lem:aux11}]
It suffices to show that $\Pr[F_2 \vert F_1]\geq 1/10$.
Indeed, this would imply that 
$\Pr[F_1] \leq 10\Pr[F_1 \cap F_2]\leq 10\Pr[F_2]$.

Let $S_1\in F_1$. Since $S_2$ and $S_1$ are independent,
it suffices to show that 
\[\Pr_{S_2\sim p^n}\bigl[(S_1,S_2)\in F_2\bigr] \geq 1/10.\]
Let $B\in\B$ such that $p(B)\leq \frac{k_{1}(B)}{4n}$.
We want to show that $k_{2}(B)\leq \frac{k_{1}(B)}{2}$ with probability at least $1/10$.
We consider two cases:
(i) if $p(B) < 1/2n$
then we bound as follows:
\begin{align*}
\Pr\Bigl[ k_{2}(B) > \frac{k_{1}(B)}{2}\Bigr]
&\leq
\Pr\Bigl[ k_{2}(B) > 0 \Bigr]\\
&\leq np(B) < 1/2 < 9/10.
\end{align*}
(ii) Else, if $p(B) \geq 1/2n$, then by the multiplicative Chernoff bound (see e.g.\cite{} page ??):
\[
\Pr\Bigl[ k_{2}(B) > \frac{k_{1}(B)}{2}\Bigr]
\leq
\Pr\Bigl[ k_{2}(B) > 2p(B)\cdot n\Bigr]
\leq
\exp\Bigl( \frac{-p(B)\cdot n}{3}\Bigr)\leq \exp(-1/6)\leq 9/10.
\]
So, conditioned on $F_1$, 
the event $F_2$ occurs with probability at least $1/10$.

\begin{proof}[Proof of Lemma~\ref{lem:aux12}]

Instead of sampling $\samp_1$ and then $\samp_2$,
we first sample $\samp=\samp_1\cup \samp_2$ and 
then partition it to $\samp_1$ and $\samp_2$ uniformly.
Now, for a fixed $\samp$ what is the probability (over the random partition)
that $F_2$ occurs?
Let $\B|_{\samp} = \{B|_{\samp} : B\in\B\}$.
It suffices to show that the event
\[F_2|_{\samp} = 
\bigl\{ \{S_1,S_2\}\text{ is a partition of $S$ into two equal parts} :
\exists {B\in\B|_{\samp}}:
  k_{i,1}>k_0 \mbox{ and } k_{i,2} \leq \frac{k_{i,1}}{2}
  \bigr\}
\]
has probability at most $\delta/10$, for every every $\samp$ 
(where the probability is over the partition of $\samp$ into $\samp_1,\samp_2$).
To analyze this we use a union bound. 
We only need to consider $B$'s in $\B|_{\samp}$ such that $k(B) > k_0$,
where $k(B) = k_{1}(B)+ k_{2}(B)$.
Fix such a $B$;
without loss of generality assume that
first $k(B)$ points out of the~$2n$
points in $S$ are in $B$. 
For every $i\leq k(B)$,
let $X_i$ denote the indicator of the event
that the $i$'th point in $S$ was drawn into $S_2$.
Set~$X=\sum_{i=1}^{k(B)}X_i$.
Note that $B$ causes $F_2|_{\samp}$ to occur if and only if $X\leq k(B)/3$.
To analyze this event define a random variable $Y=\sum_{i=1}^{k(B)}Y_i$,
where the $Y_i$'s are independent Bernoulli random variables with probability $1/2$. 
Its moment-generating function upper-bounds that of $X$, i.e. $\evp{\exp(tX)}{} \leq \evp{\exp(tY)}{}$ 
(\cite{H63}, Thm. 4), 
so we can bound the tail probability $\Pr\bigl[X\leq k(B)/3\bigr]$ using a Chernoff bound on $Y$:
\footnote{We repeatedly use this standard trick of relaxing sampling-without-replacement tail bounds based on moment-generating functions to their with-replacement counterparts in what follows.} 
\yoav{Akshay, I can see how this argument works for a single ball $B$,
  but not how it would work for a uniform bound over a set of balls.}
\akshay{Yoav, where is the issue? I believe the proof from the line ``Fix such a $B$" (before this comment inline) to ``By Sauer's Lemma" is for a fixed $B$, then the standard uniform bound applies over the Sauer class $B\mid_S$. And similarly for the other applications of Chernoff to without-replacement in this writeup.}
\shay{Here we should refer to the statement that sampling without repetitions is more concentrated.}
\yoav{This comment by shay appears verbatim in three different
  places. Should all of them be there? (instead of uncommenting
  comments, change the newcommands that create them).}
\begin{align*}
\Pr\bigl[X\leq k(B)/3\bigr]
%&\leq \Pr\bigl[Y\leq k(B)/3\bigr]\\
&=\Pr\bigl[Y- k(B)/2 < -k(B)/6\bigr]\leq
\exp\bigl(-2(1/6)^2k(B)\bigr) < \exp(-k_0/100)
\end{align*}
(where the last inequality is because $k(B) > k_0$).
By Sauer's Lemma
the number of distinct restrictions $B|_S$ is at most $\bigl(\frac{2en}{d}\bigr)^d$, and therefore 
\[\Pr[F_2]\leq \Bigl(\frac{2en}{d}\Bigr)^d\exp(-k_0/100).\]
Having $k_0\geq 200\bigl(d\ln(3n) + \ln(1/\delta) \bigr)$ yields
that this probability is at most $\delta/20$.
\end{proof}

\paragraph{Item {\it 2}.}
We now derive the second item;
let $S_1\in E_1\cap F$; we want to show that $\Pr_{S_2}[S_1S_2\in E_2]\geq 1/8$.
Fix a $B$ such that $\bigl\lvert p_{n,1}(A\vert B) - p(A\vert B) \bigr\rvert > \sqrt{\frac{k_0}{k(B)}}$.
Thus, it follows that $k(B) > k_0$, and having $S_1\in F$ implies that $p(B)\geq \frac{k_1(B)}{4n} > \frac{k_0}{4n}$.
Therefore, by basic properties\footnote{We use here that if $n\geq 2/p$ then $Z\sim Bin(n,p)$ satisfies $Z\geq np$,
with probability at least $1/4$.} of the binomial distribution it follows that $k_2(B) \geq np(B)\geq \frac{k_1(B)}{4}$ 
with probability at least $1/4$, and that conditioned on this event, by Chernoff:
\[ \bigl\lvert p_{n,2}(A \vert B) - p(A\vert B) \bigr\rvert \leq \sqrt{\frac{2\ln(2)}{k_2(B)}} < \sqrt{\frac{10}{k(B)}} \]
with probability at least $1/2$. 
(where in the last inequality we used that if $k_{2}(B)\geq k_1(B)/4$ then $k(B) \leq 5k_2(B)$).
To summarize, 
with probability at least~$\frac{1}{8}=\frac{1}{2}\cdot\frac{1}{4}$ we have that
$\bigl\lvert p_{n,2}(A \vert B) - p(A\vert B) \bigr\rvert < \sqrt{\frac{10}{k(B)}}$,
and therefore
\[
\bigl\lvert p_{n,2}(A \vert B) - p_{n,1}(A\vert B) \bigr\rvert > 
\sqrt{\frac{k_0}{k(B)}} - \sqrt{\frac{10}{k(B)}}\geq 
\sqrt{\frac{k_0}{k(B)}} - \frac{1}{2}\sqrt{\frac{k_0}{k(B)}}=
\frac{1}{2}\sqrt{\frac{k_0}{k(B)}},
\]
with probability at least $1/8$, which which implies that $S_1S_2\in E_2$ with this probability.
\end{proof}


\subsection{Proof of \ref{lem:e2}}

Let $S\in X^{2n}$. 
We need to show that
\[\Pr_{S\to S_1S_2}\Bigl[(\exists B\in \B|_S) : \bigl\lvert p_{n,1}(A \vert B) - p_{n,2}(A \vert B)  \bigr\rvert > \frac{1}{2}\sqrt{\frac{k_0}{k(B)}} \Bigr] \leq \frac{\delta}{16}.\]
To this end we show that
for every $B\in \B|_S$, 
\[\Pr_{S\to S_1S_2}\Bigl[\bigl\lvert p_{n,1}(A \vert B) - p_{n,2}(A \vert B)  \bigr\rvert > \frac{1}{2}\sqrt{\frac{k_0}{k(B)}} \Bigr] \leq \frac{\delta/16}{\lvert \B|_S\rvert},\]
which would finish the proof by a union bound.
Fix $B\in \B|_S$. 
We may assume that $k(B)> k_0$ (otherwise the above event has probability $0$). 
Denote by $k(A\cap B)$ the number of points in $S$
that are in $A\cap B$, and denote by $p_n(A\vert B) = \frac{k(A\cap B}{k(B)}$.
Since $p_{n,1},p_{n,2}$ are identically distributed, it suffices to show that
\[\Pr_{S\to S_1S_2}\Bigl[\bigl\lvert p_{n,1}(A \vert B) - p_{n}(A \vert B)  \bigr\rvert > \frac{1}{2}\cdot\frac{1}{2}\sqrt{\frac{k_0}{k(B)}} \Bigr] \leq \frac{1}{2}\cdot\frac{\delta/16}{\lvert \B|_S\rvert}.\]
To this end we show that with a sufficiently large probability, 
$k_1(B)\geq k(B)/2$, where $k_1(B)$ is the number of points in $S_1$ that belong to $B$,
and that conditioned on this typical event, the above event occurs with a sufficiently small probability.

Using the multiplicative Chernoff bound, we get 
\shay{Here we should refer to the statement that sampling without repetitions is more concentrated.}
\[ \Pr[k_1(B) < k(B)/2] \leq \exp(-k(B)/8) \leq \exp(-k_0/8).\]
Now, conditioned on that $k_1(B) \geq k(B)/2\geq k_0/2$, by Chernoff:
\[\Pr\Biggl[ \bigl\lvert p_{n,1}(A \vert B) - p_n(A \vert B) \bigr\rvert >\sqrt{\frac{2\ln(1/\delta')}{k(B)/2}}~ \Biggr\vert~ k_1(B) \geq k(B)/2\Biggr] \leq \delta',
\]
for every $\delta'$. 
Thus, plugging $\delta'$ such that $\sqrt{\frac{2\ln(1/\delta')}{k(B)/2}} = \frac{1}{4}\sqrt{\frac{k_0}{k(B)}}$
(namely $\delta = \exp(-k_0/128)$) yields that
\begin{align*}
\Pr_{S\to S_1S_2}\Bigl[\bigl\lvert p_{n,1}(A \vert B) - p_{n}(A \vert B)  \bigr\rvert > \frac{1}{2}\cdot\frac{1}{4}\sqrt{\frac{k_0}{k(B)}} \Bigr] 
&\leq  \delta' + \exp(-k_0/8)\\
&\leq  \exp(-k_0/128) + \exp(-k_0/8)\\
&\leq 2\exp(-k_0/128).
\end{align*}
Now, $\bigl\lvert \B|_S\bigr\rvert \leq \bigl(\frac{2en}{d}\bigr)^d$ by Sauer's Lemma,  and so, for $k_0\geq 200(d\ln(3n) + \ln(1/\delta))$ this probability is at most $\frac{1}{2}\cdot\frac{\delta/16}{\lvert \B|_S\rvert}$, as required.
\qed

%\newpage

%\section{Uniform convergence bounds for sepcialists.}
%
%\subsection{Uniform convergence of biases}
%
%We will use the following Lemma, which we prove in the appendix, in Section~\ref{sec:auxuc}.
%\begin{lemma}\label{lem:auxuc}
%Let $\B$ be a family of specialists of VC dimension $d$.
%Set $p_0 = \frac{100\bigl(d\ln(2n) + \ln(10/\delta)\bigr)}{n}$, 
%where $1/2\geq \delta>0$,
%and let $S=\bigl((x_i,y_i)\bigr) \sim p^n$.
%Then:
%\[
%\Pr
%\Bigl[
%\exists {B\in\B}: p_{\samp}(B) \geq p_0 \mbox{ and } p(B) \leq \frac{p_{\samp}(B)}{10}
%\Bigr]\leq \delta
%\]
%and
%\[
%\Pr
%\Bigl[
%\exists {B\in\B}: p(B) \geq p_0 \mbox{ and } p_{\samp}(B) \leq \frac{p(B)}{10}
%\Bigr]\leq \delta.
%\]
%\end{lemma}
%
%
%\appendix

\section{Proof of \ref{lem:uconeside}}\label{sec:auxuc}
\begin{proof}
Consider a double sample $S=(\samp_1,\samp_2)\sim p^{2n}$.
Let $E_1$ denote the event whose probability we want to bound:
\[
E_1 = \bigl\{S_1\in X^n : (\exists B\in\B): p_{n,1}(B) < \frac{p(B)}{4} - \frac{k_0}{n} \bigr\},
\]
and let $E_2$ denote the event:
\[
E_2 = \bigl\{S_1S_2\in X^{2n} : (\exists B\in\B): p_{n,1}(B) < \frac{p_{n,2}(B)}{2} - \frac{k_0}{n} \bigr\}.
\]
The proof follows from the next two lemmas:
\begin{lemma}\label{lem:auxuc1}
\[\Pr[E_1]\leq 2\Pr[E_2].\]
\end{lemma}
\begin{lemma}\label{lem:auxuc2}
\[\Pr[E_2]\leq \delta/2.\]
\end{lemma}
\begin{proof}[Proof of Lemma~\ref{lem:auxuc1}]
It suffices to show that $\Pr[E_2 \vert E_1]\geq 1/2$.
Indeed, this would imply that 
$\Pr[E_1] \leq 2\Pr[E_1 \land E_2]\leq 10\Pr[E_2]$.

Let $S_1\in E_1$. Since $S_2$ and $S_1$ are independent,
it suffices to show that 
\[\Pr_{S_2\sim p^n}\bigl[(S_1,S_2)\in E_2\bigr] \geq 1/2.\]
Let $B\in\B$ such that $p_{n,1}(B)< \frac{p(B)}{4} - \frac{k_0}{n}$.
We show that $p_{n,1}(B) < \frac{p_{n,2}(B)}{2} - \frac{k_0}{n}$ with probability at least~$1/2$.
For this, it suffices to show that $p_{n,2}(B)\geq \frac{p(B)}{2}$ with probability at least $1/2$;
indeed, this will imply $E_2$ by
\[p_1(B) < \frac{p(B)}{4}- \frac{k_0}{n} \leq  \frac{p_{n,2}(B)}{2}- \frac{k_0}{n}.\]
First, observe that $p(B) \geq \frac{k_0}{n}$ (because $p_{n,1}(B) \geq 0$).
Therefore, by the multiplicative Chernoff bound:
\[
\Pr\Bigl[ p_2(B) < \frac{p(B)}{2}\Bigr]
=
\Pr\Bigl[ k_2(B) < \frac{np(B)}{2}\Bigr]
\leq
\exp\Bigl( \frac{-p(B)\cdot n}{8}\Bigr) = \exp(-k_0/8)< \frac{1}{2}.
\]
So, conditioned on $E_1$, 
the event $E_2$ occurs with probability at least $1/2$.


\end{proof}

\begin{proof}[Proof of Lemma~\ref{lem:auxuc2}]

Sample $\samp=\samp_1\cup \samp_2$ and 
then partition it to $\samp_1$ and $\samp_2$ uniformly.
Let $\B|_{\samp} = \{B|_{\samp} : B\in\B\}$.
It suffices to show that the event
\[E_2|_{\samp} = 
\bigl\{ \{S_1,S_2\}\text{ is a partition of $S$ into two equal parts} :
\exists {B\in\B|_{\samp}}:
p_{n,1}(B) < \frac{p_{n,2}(B)}{2} - \frac{k_0}{n}
  \bigr\}
\]
has probability at most $\delta/2$, for every every $\samp$ 
(where the probability is over the partition of $\samp$ into $\samp_1,\samp_2$).
To analyze this we use a union bound. 
It suffices to consider $B$'s in $\B|_{\samp}$ such that $k(B) \geq k_0$,
where $k(B)$ is the number of $x\in B$ that appear in $S$.
Fix such a $B$.
Without loss of generality assume that
first $k(B)$ points out of the $2n$
points in $S$ are in $B$. 
For every $i\leq k(B)$,
let $X_i$ denote the indicator of the event
that the $i$'th point in $S$ was drawn into $S_1$.
Set~$X=\sum_{i=1}^{k(B)}X_i$.
Note that $B$ causes $E_2|_{\samp}$ to occur only if $X\leq k(B)/3$.
To analyze this event define a random variable $Y=\sum_{i=1}^{k(B)}Y_i$,
where the $Y_i$'s are independent Bernoulli random variables with probability $1/2$.
%\new{One can verify} that $\Pr[X\geq k(B)/3]\leq \Pr[Y\geq k(B)/3]$,
%and therefore it suffices to analyze the latter, 
Again note that the multiplicative Chernoff bound on $Y$ bounds the corresponding tail probability on $X$, so that:
\shay{Here we should refer to the statement that sampling without repetitions is more concentrated.}
\begin{align*}
\Pr\bigl[X\leq \frac{k(B)}{3}\bigr]&\leq 
\Pr\bigl[Y\leq \frac{k(B)}{3}\bigr]\\
&=\Pr\bigl[Y < \frac{2}{3}\cdot\frac{k(B)}{2}\bigr]\leq
\exp\Bigl(\frac{-(2/3)^2\bigl(k(B)\bigr)/2}{2}\Bigr) < \exp(-k_0/9)
\end{align*}
(where the last inequality is because $k(B) \geq k_0$).
By Sauer's Lemma
the number of distinct restrictions $B|_S$ is at most $\bigl(\frac{2en}{d}\bigr)^d$, and therefore 
\[\Pr[E_2]\leq \Bigl(\frac{2en}{d}\Bigr)^d\exp(-k_0/9).\]
Having $k_0\geq 100\bigl(d\ln(3n) + \ln(1/\delta)\bigr)$ yields
that this probability is at most $\delta/2$.
\end{proof}
\end{proof}

%\subsection{sampling without replacement}

% Transition probabilities
\newcommand{\p}{p}
\newcommand{\q}{q_{t,i}}

%$$\p=k/n$$
%$$\q=\frac{k-i}{n-t}$$

% RVs
\newcommand{\cnt}[2]{k^{#1}_{#2}}

%  bin probabilities
\newcommand{\prob}[2]{#1_{#2}}

% tail probabilities
\newcommand{\tail}[3]{{\mathbf #1}^{#2}_{#3}}

% mode
\newcommand{\mode}[1]{\mbox{mode}\left(#1\right)} 
\newcommand{\fmode}[1]{Z^{#1}_t}


Before proving Theorem~\ref{thm:Bias-Convergence} we prove a 
lemma regarding the concentration of sampling without replacement.

Let $a_1,\ldots,a_n$ be a sequence where $a_i \in
\{0,1\}$. We denote the number of 1's by $k=\sum_{i=1}^n a_i$.

We compare sampling $n$ times from $a_1^n$ $n$ with and without
replacement. We denote by $\cnt{P}{t}$ the random variable
corresponding to the number of 1's after sampling $t$ times {\em with
  replacement}, and by $\cnt{Q}{t}$ the corresponding random
variable when the sampling is done {\em without replacement}.

We deonte the marginal probability that $\cnt{P}{t}=i$ by
$\prob{P}{t,i}$ and the probability that $\cnt{Q}{t}=i$ by
$\prob{Q}{t,i}$.

Both $\prob{P}{t,i}$ and $\prob{Q}{t,i}$ have closed form formulas:
\begin{equation}\label{eqn:P-closed-form}
\prob{P}{t,i} = {t \choose i} \left(\frac{k}{n}\right)^i \left(1-\frac{k}{n}\right)^{t-i}
\end{equation}

\begin{equation}\label{eqn:Q-closed-form}
\prob{Q}{t,i}=\frac{{k \choose i}{n-k \choose t-i}}{{n \choose t}}
\end{equation}

We denote the probability of sampling a 1 when sampling with
replacement by $\p=k/n$ and the probability of sampling a 1 when
sampling with replacement at step $t$ given that the number of 1's
sampled so far is $i$ by $\q=\frac{k-i}{n-t}$

Note that if $i < \p t$ then
\[
\q=\frac{k-i}{n-t} > \frac{k-\p t}{n-t} = \frac{k(1-t/n)}{n-t} =
\frac{k(n-t)}{n(n-t)} = \frac{k}{n}=\p,
\]
and if $i > \p t$ then
\[
\q=\frac{k-i}{n-t} < \frac{k-\p t}{n-t} = \frac{k}{n}=\p.
\]
Both $P$ and $Q$ are unimodal and achieve their maximal value close to
$\p t$.

\begin{lemma}
  For sampling with replacement, define $\fmode{P} = (t+1)\frac{k}{n}-1$. If
  $i<\fmode{P}$ then $\prob{P}{t,i-1}<\prob{P}{t,i}$, if $i>\fmode{P}$
  then $\prob{P}{t,i+1}<\prob{P}{t,i}$.\newline
  For sampling with replacement, define $\fmode{Q} =(t+1)\frac{k+1}{n+2}-1$,
  If $i<\fmode{Q}$ then $\prob{Q}{t,i-1}<\prob{Q}{t,i}$, if $i>\fmode{Q}$
  then $\prob{Q}{t,i+1}<\prob{Q}{t,i}$.\newline
\end{lemma}
\begin{proof}
  The idea of the proof is to consider the ratio between the
  probabilities for consecutive values of $i$. Both distributions have
  closed form expressions:

  For sampling with replacement we have 
  \[
  \prob{P}{t,i}={t \choose i} \p^i (1-\p)^{t-i}
  \]
  From which we get that
  \[
  \frac{\prob{P}{t,i}}{\prob{P}{t,i}} = \frac{(t-i)\p}{(i+1)(1-\p)}
  \]
  Noting that the last expression is monotone decreasing with $i$ and
  solving for $\frac{\prob{P}{t,i}}{\prob{P}{t,i}}$ yields the first
  part of the lemma.\newline

  For sampling without replacement we have 
  \[
  \prob{Q}{t,i}=\frac{{k \choose i}{n-k \choose t-i}}{{n \choose t}}
  \]
  From which we get that
  \[
  \frac{\prob{Q}{t,i}}{\prob{Q}{t,i}} = \frac{(k-i)(t-i)}{(i+1)(n-k-t+i+1)}
  \]
  Again, the last expression is monotone decreasing with $i$ and
  solving for $\frac{\prob{Q}{t,i}}{\prob{Q}{t,i}}$ yields the second
  part of the lemma.
  \end{proof}


\iffalse
Let $S$ stand for $P$ or $Q$, The mode $S$ is a set of one or two
integers such that  $\mode{\prob{S}{t}} \doteq \argmax_{i} \prob{S}{t,i}$.
  
  For $0 \leq t \leq n$, for the distribution $\prob{S}{t}$ we denote the
  ``exact mode'' by a real valued number $0\leq \fmode{S} \leq t$,
  Which defines the mode as follows
  \begin{equation}\label{eqn:mode}
    \mode{\prob{S}{t}} = \begin{cases}
      0 & \mbox{if $k=0$} \\
    \lfloor \fmode{S}+1 \rfloor & \mbox{if $\fmode{S}$ is not
      an integer,} \\
      \fmode{S} \mbox{ and } \fmode{S}+1 & \mbox{if $\fmode{S}$ is an
        integer} \\
    t & \mbox{if $k=n$}
  \end{cases}
  \end{equation}
  
\fi

Next we give a recursive formulation for $P$ and $Q$. \newline
For $t=0$ we have that 
$\prob{P}{0,0}=\prob{Q}{0,0}=1$ and for all $i\neq 0$
$\prob{P}{0,i}=\prob{Q}{0,i}=0$.
For $t \geq 0$ the following recursion holds:
\[
\prob{P}{t+1,i}=\frac{k}{n} \prob{P}{t,i-1} + \left(1-\frac{k}{n}\right)\prob{P}{t,i}
\]
and
\[
\prob{Q}{t+1,i}=\frac{k-i+1}{n-t}\prob{Q}{t,i-1} + \left(1-\frac{k-i}{n-t}\right)\prob{Q}{t,i-1}
\]

For $0 \leq t \leq n$ and $0 \leq i \leq t$ we define the left and
right tails of the $P$ and $Q$ distributions as follows:
\[
\tail{L}{P}{t,i} \doteq \sum_{j=0}^i \prob{P}{t,j},\;\;\;\;\;\;
\tail{L}{Q}{t,i} \doteq \sum_{j=0}^i \prob{Q}{t,j}
\]
\[
\tail{R}{P}{t,i} \doteq \sum_{j=i}^t \prob{P}{t,j},\;\;\;\;\;\;
\tail{R}{Q}{t,i} \doteq \sum_{j=i}^t \prob{Q}{t,j}
\]
We can now state our main lemma regarding the tails of the
distribution $Q$.
\begin{lemma}
  For all $0\leq t \leq n$, let $\fmode{Q} = (t+1)\frac{k+1}{n+2}-1$,
  then 
  \begin{itemize}
    \item If $i < \lfloor \fmode{Q} \rfloor$ then $\tail{L}{Q}{t,i} < \tail{L}{P}{t,i}$
    \item If $i > \lceil \fmode{Q} \rceil$ then $\tail{R}{Q}{t,i} < \tail{R}{P}{t,i}$
    \end{itemize}
\end{lemma}

\subsection{End of section on sampling withuout replacement}


\bibliography{UnifConfCondProb,refs}
\bibliographystyle{alpha}

\end{document}
